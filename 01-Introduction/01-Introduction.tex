\section{Introduction}
\label{Sect:Intro}

%Motivation
% something more about muon collider and neutrino factory?...
Either a muon collider or a neutrino factory would require having high intensity and the phase-space volume occupied by the muon beam to be reduced before the injection into the storage ring~\cite{Geer:1998PhRvD..57.6989G}.
Due to the lifetime of the muon, the only feasible cooling technique would be the ionisation cooling \cite{Neuffer:1983jr} where muons pass through material before being re-accelerated. The net effect would be a reduction of the transverse emittance while the longitudinal energy is restored.
MICE, demonstrating for the first time this technique~\cite{Bogomilov:2019kfj}, has made the idea of those facility a concrete possibility for the future.
  
  
  
%Outline of the experiment
MICE was operated parasitically of the ISIS proton synchrotron, based at the Rutherford Appleton Laboratory (STFC).
ISIS accelerates protons up to 800\,MeV at a repetition rate of 50~Hz, while a titanium target is dipped nearly once per second in the outer halo of the proton beam.
Pions are generated in the interaction and captured inside the MICE beamline by a set of quadrupoles and dipoles~(figure~\ref{fig:BL}). Pions inside the Decay Solenoid decay to muons.
The transported beam initial emittance is altered before entering in the cooling channel by a set of adjustable diffusers providing further multiple scattering.

MICE was instrumented with a range of detectors used for particle identification and position-momentum measurement as described in this paper, used to fully characterise the beam and its evolution across the cooling channel.
