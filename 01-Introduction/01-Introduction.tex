\section{Introduction}
\label{Sect:Intro}

Stored muon beams have been proposed as the basis of a facility
capable of delivering lepton-antilepton collisions at very high
energy~\cite{Neuffer:1994bt,Palmer:2014nza} and as the source of
uniquely well-characterised neutrino 
beams~\cite{Geer:1998PhRvD..57.6989G,Bandyopadhyay:2007kx,Apollonio:2002en}.
In the majority of designs for such facilities the muons are produced
from the decay of pions created when an intense proton beam strikes a
target.
The phase-space volume occupied by the tertiary muon beam must be
reduced (cooled) before it is accelerated and subsequently injected
into the storage ring.
The time taken to cool the beam using techniques that are presently in
use at particle accelerators (synchrotron-radiation cooling
\cite{2012acph.book.....L}, laser
cooling~\cite{PhysRevLett.64.2901,PhysRevLett.67.1238,doi:10.1063/1.329218},
stochastic cooling~\cite{Marriner:2003mn}, and electron
cooling~\cite{1063-7869-43-5-R01}) are all long when compared with the
lifetime of the muon.
%The use of such techniques would therefore lead to unacceptably large
%losses through decay.
Ionization cooling~\cite{cooling_methods,Neuffer:1983jr}, in which a
muon beam is passed through a material (the absorber), where it
loses energy, and is then re-accelerated; occurs on a timescale short
compared with the muon lifetime.
Ionization cooling is therefore the only technique available to cool the muon beam at a neutrino factory or muon collider.
The international Muon Ionization Cooling Experiment (MICE)
provided the proof-of-principle demonstration of the
ionization-cooling technique~\cite{Bogomilov:2019kfj}.

MICE operated on the ISIS neutron and muon source at the STFC
Rutherford Appleton Laboratory from 2008 to 2018.  
ISIS accelerates protons to 800\,MeV at a repetition rate of
50\,Hz.
For MICE operation, a titanium target was dipped
%nearly once per second
into the halo of the proton beam with rate of 0.78\,Hz. 
Pions created in the interaction were captured in a quadrupole triplet
(see figure~\ref{fig:BL}).
A conventional beam line composed of dipole, solenoid, and quadrupole
magnets captured muons produced through pion decay and transported the
resulting muon beam to the MICE apparatus.
The momentum of the muon beam was determined by the settings of two
dipole magnets.
The emittance of the beam injected into the experiment was tuned using
a set of adjustable tungsten and brass diffusers.
The cooling cell was composed of a liquid-hydrogen or lithium-hydride
absorber placed inside a focus-coil (FC) module, sandwiched between
two scintillating-fibre trackers (TKU, TKD) placed in superconducting
solenoids (SSU, SSD).
Together, SSU, FC, and SSD formed the magnetic channel.
The MICE coordinate system is such that the $z$-axis is coincident
with the beam direction, the $y$-axis points vertically, and the
$x$-axis completes a right-handed co-ordinate system.

This paper documents the performance of the instrumentation which was
used to characterise fully the beam, its evolution across the magnetic
channel, and to quantify the properties of the liquid-hydrogen
absorber.
The instrumentation consisted of three time-of-flight detectors
(TOF0, TOF1, TOF2) discussed in section~\ref{Sect:TOF}, two 
threshold Cherenkov counters (CkovA, CkovB) discussed in
section~\ref{Sect:Ckov}, a sampling calorimeter (KL) discussed in
section~\ref{Sect:KL}, a tracking calorimeter (EMR) discussed in
section~\ref{Sect:EMR}, and the scintillating-fibre trackers
(section~\ref{Sect:Tracking}).
The instrumentation of the liquid-hydrogen absorber is discussed in
section~\ref{Sect:Absorber}.
