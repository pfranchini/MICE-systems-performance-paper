\section{Introduction}
\label{Sect:Intro}

%Motivation
% something more about muon collider and neutrino factory?...
Either a muon collider or a neutrino factory would require having high intensity and the phase-space volume occupied by the muon beam to be reduced before the injection into the storage ring~\cite{Geer:1998PhRvD..57.6989G}.
Due to the lifetime of the muon, the only feasible cooling technique would be the ionisation cooling \cite{Neuffer:1983jr} where muons pass through material before being re-accelerated. The net effect would be a reduction of the transverse emittance while the longitudinal energy is restored.
MICE, demonstrating for the first time this technique~\cite{Bogomilov:2019kfj}, has made the idea of those facility a concrete possibility for the future.

%Outline of the experiment
MICE was operated parasitically from the ISIS proton synchrotron, based at the Rutherford Appleton Laboratory (STFC).
ISIS accelerates protons up to 800\,MeV at a repetition rate of 50\,Hz. For MICE operation, a titanium target was dipped nearly once per second in the outer halo of the proton beam.
Pions were generated in the interaction and captured inside the MICE beamline~(figure~\ref{fig:BL}) by a set of triplets of quadrupoles (Q1--Q9) and dipoles (D1, D2).
Pions inside the Decay Solenoid decayed to muons. The mutual action of the bending dipoles determined the composition and the momentum spectra of the beam.
The transported beam initial emittance was altered before entering in the cooling channel by a set of adjustable diffusers providing further multiple scattering.
The cooling channel was constituted by an absorber inside a focus coil module, sandwiched between two scintillating-fibre trackers (TKU, TKD) placed in superconducting solenoids (SSU, SSD), that all together formed the magnetic channel.
MICE was instrumented with a range of detectors used for particle identification and position-momentum measurement as described in this paper: three time-of-flight detectors (TOF0, TOF1, TOF2), two Cherenkov counters (CkovA, CkovB), a sampling calorimeter (KL), a tracking calorimeter (EMR) and the already mentioned trackers.
This instrumentation was used to fully characterise the beam and its evolution across the magnetic channel.
