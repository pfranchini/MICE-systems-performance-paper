\graphicspath{{08-Track-matching/Figures/}}

\section{Global Track Matching reconstruction}
\label{Sect:TM}


\newcommand{\topmatterallplots}[4]{%
    \hspace*{-2.0cm}\includegraphics*[width=0.8\textwidth]{#1/Figures/#2/#3.png}
    \caption{#4 \label{fig:#3}}
    }

The overall detector performance can be validated by extrapolating tracks from
one detector to another and comparing the reconstructed coordinates with the 
extrapolated values.
Tracks measured in the upstream tracker are extrapolated upstream to TOF1 and TOF0, and downstream
to TKD and TOF2. Where there are materials in the beamline, the energy change on
passing through the material is estimated using the most probably energy loss. 
Material thicknesses are approximated by the on-axis thickness. 

Asymmetric effects can be introduced due to scattering from the walls of the
cooling channel as the beam is not symmetric in the channel. In order to 
minimise the effects of such scattering, only events whose projected 
trajectory is  significantly distant from the apertures are considered in this 
analysis. The following sample selection is considered:

\begin{itemize}
\item{Downstream sample:} events must be included in the downstream sample to
be considered in this analysis
\item{Aperture cut:} the projected upstream track must be within 100 mm radius 
from the beam axis at the following apertures: the upstream absorber safety window;
the upstream absorber window; the absorber centre; the downstream absorber window;
the downstream absorber safety window; the upstream edge of SSD; the Helium window
in SSD; the downstream edge of the downstream PRY aperture. This is performed
even when the lH2 absorber was not installed, for the sake of consistency and
because in some instances mounting flanges can limit the aperture and consistency.
\item{1 space point in TOF2:} the event must have exactly one space point in TOF2.
\item{Successful track extrapolation to TOF2:} the projected upstream track must
have been successfully extrapolated to TOF2
\end{itemize}

%The sample sizes are shown for data in table \ref{tab:data_cuts_summary_2_0} and
%\ref{tab:data_cuts_summary_2_1}. The equivalent MC sample sizes are listed in 
%\ref{tab:mc_cuts_summary_2_0} and \ref{tab:mc_cuts_summary_2_1}.
%
%\let\splitcell\undefined
%
\newcommand{\splitcell}[2][c]{%
\begin{tabular}[#1]{@{}c@{}}#2\end{tabular}}


%\begin{landscape}
\begin{table}
\centering
\caption{The extrapolated reconstructed data sample is listed.  Samples are listed for 3-140 and 4-140 datasets.\label{tab:data_cuts_summary_2_0}}
\scalebox{0.7}{%
\begin{tabular}[pos]{l|cccccccc}
                                                   & \splitcell{\\2017-2.7\\3-140\\None\\} & \splitcell{\\2017-2.7\\3-140\\lH2\\empty\\} & \splitcell{\\2017-2.7\\3-140\\lH2\\full\\} & \splitcell{\\2017-2.7\\3-140\\LiH\\} & \splitcell{\\2017-2.7\\4-140\\None\\} & \splitcell{\\2017-2.7\\4-140\\lH2\\empty\\} & \splitcell{\\2017-2.7\\4-140\\lH2\\full\\} & \splitcell{\\2017-2.7\\4-140\\LiH\\} \\
\hline                                            
Downstream Sample                                  &   13019  &    8688  &    9058  &   11918  &   29712  &   23726  &    8407  &   24024  \\
\hline                                            
Cooling channel aperture cut                       &    7203  &    4721  &    5166  &    6836  &   17731  &   14571  &    4935  &   14293  \\
One space point in ToF2                            &    6935  &    4506  &    4885  &    6477  &   16800  &   13817  &    4596  &   13362  \\
Successful extrapolation to TKD                    &    6935  &    4506  &    4885  &    6477  &   16800  &   13817  &    4596  &   13362  \\
Successful extrapolation to ToF2                   &    6935  &    4506  &    4885  &    6477  &   16800  &   13817  &    4596  &   13362  \\
\hline                                            
Extrapolation Sample                               &    6935  &    4506  &    4885  &    6477  &   16800  &   13817  &    4596  &   13362  \\
\hline                                            

\end{tabular}}
\end{table}
%\end{landscape}


%\begin{landscape}
\begin{table}
\centering
\caption{The extrapolated reconstructed data sample is listed.  Samples are listed for 6-140 and 10-140 datasets.\label{tab:data_cuts_summary_2_1}}
\scalebox{0.7}{%
\begin{tabular}[pos]{l|cccccccc}
                                                   & \splitcell{\\2017-2.7\\6-140\\None\\} & \splitcell{\\2017-2.7\\6-140\\lH2\\empty\\} & \splitcell{\\2017-2.7\\6-140\\lH2\\full\\} & \splitcell{\\2017-2.7\\6-140\\LiH\\} & \splitcell{\\2017-2.7\\10-140\\None\\} & \splitcell{\\2017-2.7\\10-140\\lH2\\empty\\} & \splitcell{\\2017-2.7\\10-140\\lH2\\full\\} & \splitcell{\\2017-2.7\\10-140\\LiH\\} \\
\hline                                            
Downstream Sample                                  &   27025  &   17783  &   29577  &   31257  &   14847  &    7278  &   14784  &   17138  \\
\hline                                            
Cooling channel aperture cut                       &   15238  &   10129  &   16045  &   17122  &    5633  &    2837  &    5057  &    6075  \\
One space point in ToF2                            &   14432  &    9479  &   14826  &   15774  &    5276  &    2614  &    4471  &    5372  \\
Successful extrapolation to TKD                    &   14432  &    9479  &   14826  &   15774  &    5276  &    2614  &    4471  &    5372  \\
Successful extrapolation to ToF2                   &   14432  &    9479  &   14826  &   15774  &    5276  &    2614  &    4471  &    5372  \\
\hline                                            
Extrapolation Sample                               &   14432  &    9479  &   14826  &   15774  &    5276  &    2614  &    4471  &    5372  \\
\hline                                            

\end{tabular}}
\end{table}
%\end{landscape}

%\let\splitcell\undefined
%
\newcommand{\splitcell}[2][c]{%
\begin{tabular}[#1]{@{}c@{}}#2\end{tabular}}


%\begin{landscape}
\begin{table}
\centering
\caption{The extrapolated reconstructed simulated sample is listed.  Samples are listed for 3-140 and 4-140 datasets.\label{tab:mc_cuts_summary_2_0}}
\scalebox{0.7}{%
\begin{tabular}[pos]{l|cccccccc}
                                                   & \splitcell{\\Simulated\\2017-2.7\\3-140\\None\\} & \splitcell{\\Simulated\\2017-2.7\\3-140\\lH2\\empty\\} & \splitcell{\\Simulated\\2017-2.7\\3-140\\lH2\\full\\} & \splitcell{\\Simulated\\2017-2.7\\3-140\\LiH\\} & \splitcell{\\Simulated\\2017-2.7\\4-140\\None\\} & \splitcell{\\Simulated\\2017-2.7\\4-140\\lH2\\empty\\} & \splitcell{\\Simulated\\2017-2.7\\4-140\\lH2\\full\\} & \splitcell{\\Simulated\\2017-2.7\\4-140\\LiH\\} \\
\hline                                            
Downstream Sample                                  &    8585  &    8567  &    8511  &    8624  &   18247  &   18247  &   18455  &   18553  \\
\hline                                            
Cooling channel aperture cut                       &    5112  &    4715  &    5032  &    5378  &   10884  &   10997  &   10758  &   10404  \\
One space point in ToF2                            &    4540  &    4184  &    4499  &    4820  &    9544  &    9747  &    9467  &    9117  \\
Successful extrapolation to TKD                    &    4540  &    4184  &    4499  &    4820  &    9544  &    9747  &    9467  &    9117  \\
Successful extrapolation to ToF2                   &    4540  &    4184  &    4499  &    4820  &    9544  &    9747  &    9467  &    9117  \\
\hline                                            
Extrapolation Sample                               &    4540  &    4184  &    4499  &    4820  &    9544  &    9747  &    9467  &    9117  \\
\hline                                            

\end{tabular}}
\end{table}
%\end{landscape}


%\begin{landscape}
\begin{table}
\centering
\caption{The extrapolated reconstructed simulated sample is listed.  Samples are listed for 6-140 and 10-140 datasets.\label{tab:mc_cuts_summary_2_1}}
\scalebox{0.7}{%
\begin{tabular}[pos]{l|cccccccc}
                                                   & \splitcell{\\Simulated\\2017-2.7\\6-140\\None\\} & \splitcell{\\Simulated\\2017-2.7\\6-140\\lH2\\empty\\} & \splitcell{\\Simulated\\2017-2.7\\6-140\\lH2\\full\\} & \splitcell{\\Simulated\\2017-2.7\\6-140\\LiH\\} & \splitcell{\\Simulated\\2017-2.7\\10-140\\None\\} & \splitcell{\\Simulated\\2017-2.7\\10-140\\lH2\\empty\\} & \splitcell{\\Simulated\\2017-2.7\\10-140\\lH2\\full\\} & \splitcell{\\Simulated\\2017-2.7\\10-140\\LiH\\} \\
\hline                                            
Downstream Sample                                  &   17810  &   18031  &   18188  &   18259  &    8843  &    9029  &    9155  &    9294  \\
\hline                                            
Cooling channel aperture cut                       &   10289  &   10071  &    9449  &    9906  &    3424  &    3393  &    3227  &    3337  \\
One space point in ToF2                            &    9014  &    8766  &    8219  &    8577  &    2941  &    2926  &    2772  &    2861  \\
Successful extrapolation to TKD                    &    9014  &    8766  &    8219  &    8577  &    2941  &    2926  &    2772  &    2861  \\
Successful extrapolation to ToF2                   &    9014  &    8766  &    8219  &    8577  &    2941  &    2926  &    2772  &    2861  \\
\hline                                            
Extrapolation Sample                               &    9014  &    8766  &    8219  &    8577  &    2941  &    2926  &    2772  &    2861  \\
\hline                                            

\end{tabular}}
\end{table}
%\end{landscape}

%\let\splitcell\undefined

%\begin{figure}[!tbh]
%    \centering
%    \topmatterallplots{08-Track-matching}{/}{global_through_residual_tof1_x}
%    {Residual horizontal (x) position in TOF1 of tracker tracks following extrapolation from TKU. \label{fig:tof1_extrapolated_x}}
%\end{figure}

%\begin{figure}[!tbh]
%    \centering
%    \topmatterallplots{08-Track-matching}{compare_globals}{global_through_residual_tof1_y}
%    {Residual vertical (y) position in TOF1 of tracker tracks following extrapolation from TKU. \label{fig:tof1_extrapolated_y}}
%\end{figure}
%
%\begin{figure}[!tbh]
%    \centering
%    \topmatterallplots{08-Track-matching}{compare_globals}{global_through_residual_tof0_t}
%    {Residual TOF0 time of the extrapolated track. Track trajectories were drawn from TKU, while the track times were
%    drawn from TOF1 with appropriate offsets for time-of-flight from TKU to TOF1 considered. \label{fig:tof0_extrapolated_t}}
%\end{figure}

%A typical residual plot is shown in Fig.~\ref{fig:tof1_extrapolated_x} for the extrapolated position following extrapolation to TOF1.
%In general the width of the distributions are comparable between MC and data. Where the diffuser is in place for higher emittance beams, the extrapolation goes through the diffuser material %so the residuals are wider, owing to the increased scattering from the diffuser.

%The extrapolated position following extrapolation to TOF1 is shown in Fig.~\ref{fig:tof1_extrapolated_x} and \ref{fig:tof1_extrapolated_y}.

%The time-of-flight residual in data shows a systematic offset from 0 and relative to the
%MC, as in Fig.~\ref{fig:tof0_extrapolated_t}. The offset from 0 gets worse for higher emittance beams. It is thought to be
%an intrinsic property of the beam; muons that are scattered in materials between 
%the tracker and the TOF have systematically shorter path lengths than the 
%extrapolated trajectories, resulting in systematically longer extrapolated time
%of flight. The MC reconstruction is known to have issues, as evidenced by the 
%discrepancy in slab $dt$ for TOF0 and TOF1.

%\textcolor{red}{Plot momentum vs dt for TOF01 and TOF12}

%\begin{figure}[!tbh]
%    \centering
%    \topmatterallplots{08-Track-matching}{compare_globals}{global_through_residual_tkd_tp_x}
%    {Residual $x$ position of TKU tracks extrapolated to TKD, as compared to the tracks in TKD. \label{fig:tkd_extrapolated_x}}
%\end{figure}
%
%\begin{figure}[!tbh]
%    \centering
%    \topmatterallplots{08-Track-matching}{compare_globals}{global_through_residual_tkd_tp_y}
%    {Residual $y$ position of TKU tracks extrapolated to TKD, as compared to the tracks in TKD. \label{fig:tkd_extrapolated_y}}
%\end{figure}
%
%\begin{figure}[!tbh]
%    \centering
%    \topmatterallplots{08-Track-matching}{compare_globals}{global_through_residual_tkd_tp_px}
%    {Residual $p_x$ of TKU tracks extrapolated to TKD, as compared to the tracks in TKD. \label{fig:tkd_extrapolated_px}}
%\end{figure}
%
%\begin{figure}[!tbh]
%    \centering
%    \topmatterallplots{08-Track-matching}{compare_globals}{global_through_residual_tkd_tp_py}
%    {Residual $p_y$ of TKU tracks extrapolated to TKD, as compared to the tracks in TKD. \label{fig:tkd_extrapolated_py}}
%\end{figure}
%
%\begin{figure}[!tbh]
%    \centering
%    \topmatterallplots{08-Track-matching}{compare_globals}{global_through_residual_tkd_tp_p}
%    {Residual $p_{tot}$ of TKU tracks extrapolated to TKD, as compared to the tracks in TKD. \label{fig:tkd_extrapolated_p}}
%\end{figure}

%Small misalignments between TKU extrapolated tracks and TKD are observed, 
%indicated by the offset of transverse variables from 0, shown in 
%Fig.~\ref{fig:tkd_extrapolated_x} and \ref{fig:tkd_extrapolated_y}. There are known, 
%uncorrected misalignments in the detector system and there are expected to be 
%additional misalignments in the magnets which could lead to these offsets.

%The total momentum shows discrepancy between TKU and TKD of about 1 MeV/c. This
%is consistent with the systematic offset in the tracker momentum resolution 
%shown in Fig.~\ref{fig:tku_resolution} and \ref{fig:tkd_resolution}. It is
%interesting to note that the level of agreement between MC and data varies
%on a setting-by-setting basis in a statistically significant manner. Agreement
%is better for the settings where the liquid hydrogen windows were installed.

%\begin{figure}[!tbh]
%    \centering
%    \topmatterallplots{08-Track-matching}{compare_globals}{global_through_residual_tof2_t}
%    {Residual time of TKU tracks extrapolated to TOF2, as compared to the time measured in TOF2. The track times were
%    drawn from TOF1 with appropriate offsets for time-of-flight from TKU to TOF1 considered.}
%\end{figure}
%
%\begin{figure}[!tbh]
%    \centering
%    \topmatterallplots{08-Track-matching}{compare_globals}{global_ds_residual_tof2_x}
%    {Residual $x$ position of TKD tracks extrapolated to TOF2, as compared to the position measured in TOF2.}
%\end{figure}
%
%\begin{figure}[!tbh]
%    \centering
%    \topmatterallplots{08-Track-matching}{compare_globals}{global_ds_residual_tof2_y}
%    {Residual $y$ position of TKD tracks extrapolated to TOF2, as compared to the position measured in TOF2.}
%\end{figure}

%Further small misalignments are observed in the position residuals between TOF2
%and tracks extrapolated from TKD. This is attributed to alignment issues.

%TOF2 exhibits a significant offset from the extrapolated track.

% Not for now
%\subsection{Beam based magnet alignment}
%\label{SubSect:TM_Magnet_Alignment}
