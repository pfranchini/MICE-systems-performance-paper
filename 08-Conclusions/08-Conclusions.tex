\graphicspath{{80-Conclusions/Figures/}}

\section{Summary and conclusions}
\label{Sect:Conclusions}

A complete set of particle detectors has permitted the full characterisation and study of the evolution of the phase space of a muon beam through a section of a cooling channel in the presence of liquid hydrogen and lithium hydride absorbers, leading to the first measurement of ionization cooling.
The PID performance of the detectors is summarised in table~\ref{tab:pid1} and table~\ref{tab:pid2} and is fully compatible with the specification of the apparatus~\cite{NOTE21}.

\begin{table}[htb!]
	\caption{Summary of the performance of the MICE PID detectors.}
  \begin{center}
	\begin{tabular}{|c|c|c|}
   	\hline
	  \textbf{Detector}              & \textbf{Characteristic}            & \textbf{Performance} \\
		\hline
    Time-of-Flight        & time resolution           & 0.10\,ns    \\
%   Cherenkov             & ??                        & ??          \\
    KLOE-Light            & muon PID efficiency       & ~99\%       \\
    Electron Muon Ranger  & electron PID efficiency   & 98.6\%      \\
%   Trackers              & track finding efficiency  & $>$98\%     \\
    \hline
  \end{tabular}
	\label{tab:pid1}
  \end{center}
\end{table}


\begin{table}[htb!]
	\caption{Summary of the MICE PID detector performance for different beam settings.}
  \begin{center}
  \begin{tabular}{|c|ccc|cc|cc|cc|cc|}
    \hline
  &

  \multicolumn{3}{c|}{\textbf{KL efficiency}} &
  \multicolumn{2}{c|}{\textbf{EMR efficiency}} &
  \multicolumn{6}{c|}{\textbf{Track finding efficiency}} \\ 
  \hline
  \multirow{2}{*}{Momentum} &
  \multirow{2}{*}{electrons} &
  \multirow{2}{*}{muons} &
  \multirow{2}{*}{pions} &
  \multirow{2}{*}{electrons} &
  \multirow{2}{*}{muons} &
  \multicolumn{2}{c|}{3 mm} &
  \multicolumn{2}{c|}{6 mm} &
  \multicolumn{2}{c|}{10 mm} \\  
             &      &      &      &       &      & US   & DS   & US   & DS   & US   & DS   \\ \hline
\textbf{140\,MeV/$c$} & 95\% & 97\% & n.a.   & 98\%  & 35\% &        &      & 98\%   & 99\% & 98\% & 99\% \\ \hline
\textbf{170\,MeV/$c$} & 95\% & 99\% & 89\% & 99\%  & 99\% &        &      &          &        &        &      \\ \hline
\textbf{200\,MeV/$c$} & 94\% & 99\% & 95\% & 100\% & 99\% & 99\% & 96\% & 99\% & 96\% &        &      \\ \hline
\textbf{240\,MeV/$c$} & 96\% & 99\% & 97\% & 99\%  & 99\% &        &      &          &        &        &      \\ \hline
\textbf{300\,MeV/$c$} & 95\% & 99\% & 98\% & n.a.    & 99\% &        &      &          &        &        &      \\ \hline
  \end{tabular}
	\label{tab:pid2}
  \end{center}
\end{table}
All the different elements of the MICE instrumentation have been used to characterise the beam and the measurement of the cooling performance for a different variety of beam momenta, emittance, and absorbers. The measurement of the physical properties of the liquid hydrogen absorber have been fully described here.
The experiment has thus demonstrated a technique critical for a muon collider and a neutrino factory and brings those facilities one step closer.
