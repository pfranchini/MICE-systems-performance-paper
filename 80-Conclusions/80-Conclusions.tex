\graphicspath{{80-Conclusions/Figures/}}

\section{Conclusions}
\label{Sect:Conclusions}

A complete set of particle detectors has permitted the full characterisation and study of the evolution of the phase space of a muon beam through a section of a cooling channel, leading to the first measurement of ionization cooling with a unique level of precision and detail.
The PID performance of the detectors is summarised in table~\ref{tab:pid}.


\begin{table}[htb!]
	\centering
	\begin{tabular}{c|c|c}
	  Detector              & characteristic            & performance \\
		\hline
    Time-of-Flight        & time resolution           & 0.10\,ns \\
    Cherenkov             & ?? &\\
    KLOE-Light            & ?? &\\
    Electron Muon Ranger  & electron PID efficiency   & 98.6\%  \\
    Trackers              & track finding efficiency  & $>$98\% \\
  \end{tabular}
	\caption{Summary of the MICE detectors PID.}
	\label{tab:pid}
\end{table}

All the different elements of the MICE instrumentation have been used to characterise the beam and the measurement of the cooling performance for a different variety of beam momenta, emittance, and absorbers.
The experiment has thus demonstrated a technique critical for a muon collider and a neutrino factory and therefore brings those facilities one step closer.
