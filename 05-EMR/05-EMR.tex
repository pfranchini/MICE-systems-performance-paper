\graphicspath{{05-EMR/Figures/}}

\section{Electron Muon Ranger}
\label{Sect:EMR}

The main purpose of the Electron-Muon Ranger (EMR) was to identify
positrons arising from muon decay within the apparatus.  
The EMR was a fully-active scintillator
detector~\cite{2016JInst..11T10007} with a granularity that allowed
for track reconstruction.
The EMR consisted of extruded triangular scintillator bars arranged in
planes.
One plane contained 59 bars and covered an area of 1.27\,m$^2$.
Figure~\ref{fig:EMR} the bar cross section and the arrangement of the
bars in a plane.
Triangular bars were chosen so that tracks moving parallel to the
detector axis would not travel along the gaps between bars. 
Planes were mounted such that the angle between successive planes was
$90^{\circ}$ so that hits in neighbouring planes defined horizontal
and vertical interaction coordinates.
A single ``X--Y module'' was composed of an orthogonal pair of
planes.
The scintillation light was collected by using a wave-length shifting
(WLS) fibre glued inside each bar.
At each end, the WLS fibre was coupled to clear fibres that
transported the light to a PMT.
All the WLS fibres from one edge of a plane were read out using a
single-anode PMT so that an integrated charge measurement could be
used to determine the energy deposited in the plane.
The signals from the fibres emerging from the other edge of the plane
were recorded individually using multi-anode PMTs (MAPMTs). 
The full detector was composed of 24 X--Y modules giving a total active 
volume of approximately~1\,m$^3$.
\begin{figure}[htb!]
  \begin{center}
    \includegraphics[width=0.465\columnwidth]{EMR1.png}
    \hfill
    \includegraphics[width=0.515\columnwidth]{EMR2.jpg}
  \end{center}
  \caption{
    Drawing of one EMR plane (top left), cross section of the
    arrangement of 3 bars and their wavelength shifting fibres (bottom
    left) and drawing of the full detector and its supporting
    structure (right).
  }
  \label{fig:EMR}
\end{figure}

Measurements of the performance of the completed detector demonstrated
an efficiency per plane
of~$99.73\pm0.02$\,\%~\cite{2016JInst..11T10007,Drielsma:2017doj}.
The level of crosstalk was within acceptable values for the type of
multi-anode photomultiplier used, with an average of $0.20\pm0.03$\%
probability of occurrence in adjacent channels and a mean amplitude
equivalent to $4.5\pm0.1$\,\% of the primary signal intensity.
Only four dead bars were identified.

The primary purpose of the EMR was to distinguish between muons and
their decay products, identifying muons that have crossed the entire
magnetic channel.
Muons and electrons exhibited distinct behaviours in the detector.
A muon followed a single straight track before either stopping or
exiting the scintillating volume.
Electrons showered in the lead of the KL and created a broad cascade
of secondary particles.
Two main geometric variables, the plane density and the shower spread,
were used to differentiate them.
The detector was capable of identifying electrons with an efficiency
of 98.6\,\%, providing a purity for the MICE beam that exceeds
99.8\,\%.
The EMR also proved to be a powerful tool for the reconstruction of
muon momenta in the range
100--280\,MeV/$c$~\cite{2015JInst..10P2012A}.  \\

\noindent\textbf{Performance} \\
\noindent
A full description of the detector and the reconstruction algorithms
used may be found in reference~\cite{2015JInst..10P2012A}.
Here the performance of the EMR detector over the course of the
experiment is summarised.

To maximise beam transmission to the EMR and to increase the range of
muons within the detector the MICE beamline was set to deliver a
nominal momentum of approximately 400\,MeV/$c$.
In this configuration the beamline produced pions and muons in
comparable quantities and positrons.
Time-of-flight between TOF1 and TOF2 was used to identify particle
species and only particles compatible with the muon hypothesis were
included in the analysis.
Particles entering the muon sample had a momentum larger than
350\,MeV/$c$ at the upstream surface of TOF2 and were expected to
cross both TOF2 and the KL and penetrate the EMR.
$99.62\pm0.03\,\%$ of the particles entering TOF2 were observed to produce
hits in the EMR.
The small inefficiency may be attributed to pions in the muon sample
that experienced hadronic interactions in the KL.
If hits were produced in the detector, an $(x,y)$ pair, defining a
space point, was reconstructed $98.56\pm0.06\,\%$ of the time.

To evaluate the efficiency of the scintillator planes, only the muons
that penetrated the entire detector were used.
Muons were selected which produced a hit in the most downstream plane.
For these events a hit was expected in at least one bar in each plane
on its path.
In $3.26\pm0.02\,\%$ of cases a plane traversed by a muon did not
produce a signal in the MAPMT, the mode of the hit-multiplicity
distribution per plane was one, and the probability that the track was
not observed in the SAPMT was $1.88\pm0.01\,\%$. \\

\noindent\textbf{Electron rejection} \\
\noindent
A broad range of beamline momentum settings was used to characterise
the electron-rejection efficiency.
Particle species was characterised upstream of the EMR using the
time-of-flight between TOF1 and TOF2.
For each momentum setting, a Gaussian fit was carried out to determine
the position of the muon and positron time-of-flight peaks.
Events which fell within {\color{red} XX standard deviations} of the
central value of the muon time-of-flight peak were were accepted into
a muon-template sample while events which fell within {\color{red} XX
standard deviations} of the positron peak formed the positron-template
sample. 
Particles with a time-of-flight larger than the upper limit of the
muon sample were either pions or slow muons and were rejected.

The hits produced by a muon in the EMR appear as a clean straight
track.
By contrast, an electromagnetic shower was initiated by the passage of
a positron through the lead of the KL.
This resulted in a disjointed and spread-out hit signature for
positrons in the EMR.
Two particle-identification variables were defined based on these
distinct characteristics.
The first is the plane density, $\rho_p$
\begin{equation}
  \rho_p = \frac{N_p}{Z_p+1},
\end{equation}
where $N_p$ is the number of planes hit and $Z_p$ the number of the
most downstream plane~\cite{2015JInst..10P2012A}.
A muon deposits energy in every plane it crosses until it stops,
producing a plane density close to one.
A positron shower contains photons that may produce hits deep inside
the fiducial volume without leaving a trace on their path, reducing
the plane density.
The second variable is the normalised $\hat{\chi}^2$ of the fitted
straight track given by
\begin{equation}
  \hat{\chi}^2=\frac{1}{N-4}\sum_{i=1}^{N}\frac{\text{res}_{x,i}^2+\text{res}_{y,i}^2}{\sigma_x^2+\sigma_y^2},
\end{equation}
where $N$ is the number of space points (one per bar hit),
$\text{res}_{q,i}$ the residual of the space point with respect to the
track in the $qz$ projection and $\sigma_q$ the uncertainty on the
space point in the $qz$ projection, $q=x,\,y$~\cite{Drielsma:thesis}.
This quantity represents the transverse spread of the hits produced by
the particle in the EMR.
A muon produced a single track and was expected to have a
$\hat{\chi}^2$ close to one, while an electron shower was expected to
produce a larger value.
The two discriminating variables can be combined to form a statistical
test on the particle hypothesis. 
Dense and narrow events will be tagged as muons while non-continuous
and wide electron showers will not.  
The quality of a test statistic may be characterised in terms of the
fraction, $\alpha$, of the the muon sample that is rejected, and the 
contamination, $\beta$, the fraction of the electron sample that is
selected. 

%The downstream tracker allows the reconstruction of particle momentum
%upstream of the EMR. 
%To assess the influence of momentum on contamination and loss, their
%values were calculated in 10\,MeV/$c$ momentum bins in the range
%100--300\,MeV/$c$.

The momentum of the particles was measured by the downstream tracker 
and this information used to determine the momentum dependence of the 
contamination and loss in the range 100--300\,MeV/$c$.
Figure~\ref{fig:emr_pid_mom} shows the loss, $\alpha$, and the
contamination, $\beta$, as a function of the momentum measured in
TKD.
The loss statistic increases towards low muon momentum.
This is due both to an increase in the decay probability between TOF2
and the EMR and a decrease in the number of muons that cross the KL to
reach the EMR. 
\begin{figure}
  \begin{center}
    \includegraphics[width=0.68\columnwidth]{pid_mom.pdf}
  \end{center}
  \caption{
    Percentage of electron contamination, $\beta$, and muon loss,
    $\alpha$, for different ranges of momentum measured in the
    downstream tracker, $p_d$.
    The error bars are based on the statistical uncertainty in a bin,
    and the bin width set by the resolution of the measurement.
  }
  \label{fig:emr_pid_mom}
\end{figure}
