\graphicspath{{03-Ckov/Figures/}}

\section{Cherenkov Detectors}
\label{Sect:Ckov}

\subsection{Introduction}
\label{SubSect:Ckov_Intro}
% Lucien Cremaldi (and David Sander) reviewed this

The MICE threshold Cherenkov detectors, measuring velocity, were primarily designed to provide $\pi$-$\mu$ separation in the higher momentum ranges, where time-of-flight was not sufficient for conclusive particle identification.

In order to provide separation over a large range of momenta, two high-density silica aerogel Cherenkov detectors (CkovA and CkovB) with refractive indices $n$=1.07 and $n$=1.12 were used.
Light was collected in each counter by four eight-inch UV-enhanced PMTs and recorded by CAEN V1731 FADCs (500\,MS/s)~\cite{NOTE473}.
The two detectors were placed directly one after another in the beamline, located just after the first TOF counter. In~figure~\ref{fig:ckov1} an exploded view of one detector is shown.

CkovA and CkovB thresholds provided different responses in four distinct momentum ranges, i.e. in the 200\,MeV/$c$ beams, pions were below the threshold which would fire the detector for both CkovA and CkovB whereas muons were above the threshold only for CkovB, while for the 240\,MeV/$c$ beams, pions were above the threshold for CkovB while muons were above for both CkovA and CkovB.
This can be used to distinguish particle species.
Below the CkovB muon threshold of about 217.9\,MeV/$c$, where there was no separation, the TOFs provided good separation, whereas the momentum range above the CkovA pion threshold 367.9\,MeV/$c$ was outside of the MICE running parameters~\cite{NOTE473}.
For unambiguous identification of particle species, the Cherenkov detectors would need a momentum measurement from the MICE tracker.

\begin{figure}[htb!]
  \begin{center}
    \includegraphics[width=0.6\columnwidth]{./03-Ckov/Figures/Ckov_fix.png}
    \caption{MICE aerogel Cherenkov counter blowup: a)~entrance window, b)~mirror, c)~aerogel mosaic, d)~acetate window, e)~GORE reflector panel, f)~exit window and g)~eight-inch PMT in iron shield.}
    \label{fig:ckov1}
  \end{center}
\end{figure}

\subsection{Performance}
\label{SubSect:Ckov_Performance}

Beams with a wide momentum range have been used to cover the full spectra of particles that could have been measured by the Cherenkov detectors.
The asymptotic light yield (for $\beta$=$v$/$c$=$1$) in each counter was measured using the electron peaks, resulting in $16\pm1$ photoelectrons (NPE) in CkovA and $19\pm1$ in CkovB.

%\begin{table}
%  \begin{center}
%    \begin{tabular}{c|c|c|c|c}
%       Run ID & Date & Nominal momentum [MeV/$c$] & Spills & Triggers \\
%  		\hline
%       10488  & 12/12/2017 & 140 & 3777 & 300269 \\
%       10496  & 12/12/2017 & 170 & 4180 & 371037 \\
%       10391  & 03/12/2017 & 200 & 2146 & 240033 \\
%       10419  & 04/12/2017 & 240 & 2932 & 328062 \\
%       10304  & 29/11/2017 & 300 & 2502 & 305363 \\
%       10221  & 23/11/2016 & 300 & 4493 & 560119 \\
%       10519  & 15/12/2017 & 400 & 4316 & 506384 \\
%    \end{tabular}
%   \caption{Summary of the data sets used to visualize the activation curve of the MICE Cherenkov detectors.}
%   \label{tab:ckov}
%  \end{center}
%\end{table}


The photoelectron yields versus $\beta\gamma$ in CkovA and CkovB are displayed in figure~\ref{fig:ckov_betagamma}, where the background NPE has not been included in the plot in order to highlight the activated part.

%The photoelectron yields versus momentum for muons and pions in CkovA and CkovB are displayed in figure~\ref{fig:ckov2}, using the time of flight between TOF0 and TOF1 to select the species and to estimate the momentum.

%The distributions $N(P)$ of NPEs as a function of the momentum P were fitted with the function

%\begin{equation}
%N(P) = N_0 + N_{\beta=1}\left(1-\left(\frac{P_0}{P}\right)^2\right)
%\end{equation}
%where $N_0$ is the number of background photoelectrons, $N_{\beta=1}$ is the asymptotic light yield and $P_0$ is the momentum threshold.

%\begin{figure}[htb!]
%  \begin{center}
%    \includegraphics[width=0.85\columnwidth]{./03-Ckov/Figures/Ckov_photoelectrons_vs_P.pdf}
%    \caption{Photoelectron yields versus momentum for muons and pions in CkovA and CkovB with the superimposed fitted activation functions. The error bars correspond to the standard deviation for each momentum bin.}
%    \label{fig:ckov2}
%  \end{center}
%\end{figure}

%\begin{figure}[htb!]
%  \begin{center}
%    \includegraphics[width=0.85\columnwidth]{./03-Ckov/Figures/scatter_activation.png}
%    \caption{Photoelectron yields versus momentum for muons and pions in CkovA and CkovB with the superimposed fitted activation functions.}
%    \label{fig:ckov2}
%  \end{center}
%\end{figure}

\begin{figure}[htb!]
  \begin{center}
    \includegraphics[width=0.90\columnwidth]{./03-Ckov/Figures/scatter_betagamma_logo.png}
    \caption{Photoelectron yields versus $\beta\gamma$ in CkovA and CkovB.}
    \label{fig:ckov_betagamma}
  \end{center}
\end{figure}

The refractive indices than can be calculate from the turn on point ($n =  \sqrt{1+\frac{1}{\beta_0^2\gamma_0^2}}$) are compatible with the nominal values quoted before.

%The observed muon thresholds were $267.3\pm18.1$ and $219.4\pm14.5$\,MeV/$c$, while for pions were $332.2\pm38.4$\,MeV/$c$ and $295.9\pm95.9$\,MeV/$c$, respectively in CkovA and CkovB. The $NPE_{\beta=1}$ values were generally lower than the values predicted in a previous analysis~\cite{NOTE473}: this was mostly due to TOF0 acting as a pre-shower radiator.

%In figure~\ref{fig:ckov3} was shown the typical photoelectron spectra for muons and pions well above the threshold. The expected Poisson-like distribution receives tails from the electromagnetic showers and from secondary electrons coming from interactions in TOF0 and in the aerogel itself.

%\begin{figure}[htb!]
%  \begin{center}
%    \includegraphics[width=0.85\columnwidth]{./03-Ckov/Figures/Ckov_photoelectrons_spectra.pdf}
%    \caption{Photoelectron spectra in CkovA (left) and CkovB (right) for muons (green) and pions (blue) above (continuous line) and below (dashed line) the threshold. The distributions are normalised to the same area.}
%    \label{fig:ckov3}
%  \end{center}
%\end{figure}
